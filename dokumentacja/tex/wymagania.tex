\chapter{Wymagania}



\section{Wymagania funkcjonalne}

%\begin{table}[!h]
%\label{tab:fr}
  %\begin{center}
  \begin{longtable}{| p{.20\textwidth} || p{.70\textwidth} |} 
\hline
\textbf{Identyfikator wymagania} & \textbf{Opis wymagania} \\ \hline
FR1 & System umożliwia wyświetlenie listy dostępnych do pobrania wyników badań medycznych \\ \hline
FR2 & System umożliwia pobranie zanonimizowanych wyników badań medycznych
 \\ \hline
FR3 & System umożliwia pobranie statystyk wykonanych badań medycznych
 \\ \hline
  \end{longtable} 
  %\end{center}
%\end{table}

\section{Wymagania niefunkcjonalne}

%\begin{table}[!h]
%\label{tab:nfr}
  %\begin{center}
  \begin{longtable}{| p{.20\textwidth} || p{.70\textwidth} |} 
\hline
\textbf{Identyfikator wymagania} & \textbf{Opis wymagania} \\ \hline
NFR1 & Współbieżne oprogramowanie realizujące część serwerową \\ \hline
NFR2 & Uszkodzenie węzła, nie powoduje zatrzymania pracy systemu
 \\ \hline
NFR3 & Realizacja usługi w trakcie awarii w czasie obsługi
 \\ \hline
NFR4 & Możliwość ponownego wpięcia węzła, z którym utracono łączność
 \\ \hline
NFR5 & Odporność na próbę wpięcia wrogiego, nieuprawnionego węzła
 \\ \hline
NFR6 & Zarządzanie zasobami transparentnie dla oprogramowania klienckiego
 \\ \hline
NFR7 & Zapewnienie poziomu redundancji danych równego 2
 \\ \hline
NFR8 & Maksymalna objętość przechowywanych wszystkich danych równa 1GB
 \\ \hline
NFR9 & Dane o pacjentach oraz badaniach przechowywane w relacyjnej bazie danych PostrgreSQL
 \\ \hline
NFR10 & Wyniki badań medycznych przechowywane w plikach formatu .xml lub .bmp na serwerze danych
 \\ \hline
NFR11 & Uruchamianie i zamykanie części serwerowej jednokrotnym wywołaniem skryptu dowolnym węźle
 \\ \hline
NFR12 & Liczba jednocześnie obsłużonych użytkowników równa 100
 \\ \hline
NFR13 & System uruchamiany w środowisku Linux Ubuntu 14.04 LTS
 \\ \hline
NFR14 & Dwa serwery danych, każdy posiada procesor z minimum czterema wątkami sprzętowymi oraz dyskiem twardym o pojemności min. 20 GB
 \\ \hline
NFR15 & Klaster trzech serwerów (nie wliczając serwerów danych), każdy posiada     procesor z minimum dwoma wątkami sprzętowymi
 \\ \hline
  \end{longtable}  
  %\end{center}
%\end{table}