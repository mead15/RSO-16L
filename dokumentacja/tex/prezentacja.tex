\chapter{Prezentacja systemu}
\label{ap:2}

\section{Opis prezentacji systemu}

\begin{enumerate}
	\item Start aplikacji.
	\begin{enumerate}[label*=\arabic*.]
		\item Uruchomienie aplikacji poprzez wywołanie komendy ./ServerDaemon master na którymkolwiek z węzłów warstwy zewnętrznej bądź wewnętrznej.
		\item Sprawdzenie uruchomienia na innych węzłach. 
	\end{enumerate}
	\item Pobranie pliku konfiguracyjnego.
	\begin{enumerate}[label*=\arabic*.]
		\item Wykonanie kopii starego pliku konfiguracyjnego.
		\item Wywołanie klienta bez parametrów.
		\item Porównanie starego pliku z nowo pobranym.
	\end{enumerate}
	\item Połączenie z serwerami alternatywnymi.
	\begin{enumerate}[label*=\arabic*.]
		\item Odłączenie serwera widniejącego na pierwszej pozycji w pliku konfiguracyjnym klienta (klient powinien wybierać kolejne serwery).
		\item Wykonanie dowolnego zapytania klienta.
	\end{enumerate}
	\item Pobieranie danych przez klienta.
	\begin{enumerate}[label*=\arabic*.]
		\item Pobranie listy badań (uruchomienie klienta z parametrem ‘list’).
		\item Pobranie badania o podanym id (uruchomienie klienta z parametrem ‘get’) i sprawdzenie anonimowości danych.
		\item Wyszukiwanie badań z filtrowaniem (uruchomienie klienta z parametrem ‘search’ i filtrami: nazwa ‘*’, kraj ‘Polska’, płeć ‘M’, rasa ‘*’, minimalny wiek ‘*’, maksymalny wiek ‘*’).
		\item Umyślne wywołania klienta dla błędnych zapytań. 
	\end{enumerate}
	\item Odporność na odłączanie węzłów.
	\begin{enumerate}[label*=\arabic*.]
		\item Uruchomienie skryptu pobierającego za pomocą klienta listy badań i zapisującego ją w pliku. 
		\item Odłączenie węzła warstwy zewnętrznej i węzła warstwy wewnętrznej.
		\item Ponowne uruchomienie skryptu i porównanie nowo pobranej listy z wcześniejszą.
	\end{enumerate}
	\item Odporność na wymianę węzłów.
	\begin{enumerate}[label*=\arabic*.]
		\item Uruchomienie usługi z dwoma serwerami warstwy zewnętrznej. 
		\item Uruchomienie skryptu pobierającego za pomocą klienta listy badań i zapisującego ją w pliku.
		\item Dołączenie trzeciego węzła warstwy zewnętrznej.
		\item Odłączenie węzłów działających przed podłączeniem trzeciego węzła.
		\item Ponowne uruchomienie skryptu i porównanie nowo pobranej listy z wcześniejszą.
	\end{enumerate}
	\item Spójność danych.
	\begin{enumerate}[label*=\arabic*.]
		\item Wprowadzenie, przez klienta wrzucającego dane, badania zawierającego plik. 
		\item Uruchomienie klienta z parametrem list i upewnienie się o dodaniu badania.
		\item Odłączenie serwera danych skomunikowanego z klientem wrzucającym dane.
		\item Ponowne wywołanie listy badań i upewnienie się o tym, że wcześniej dodane badanie znajduje się na liście.
	\end{enumerate}
	\item Elekcja.
	\begin{enumerate}[label*=\arabic*.]
		\item Znalezienie koordynatorów warstwy wewnętrznej i zewnętrznej.
		\item Odłączenie koordynatorów. 
		\item Sprawdzenie dowolnej funkcji systemu.  
	\end{enumerate}
	\item Elekcja.
	\begin{enumerate}[label*=\arabic*.]
		\item Uruchomienie skryptu wywołującego zapytanie (pobierające plik)  klienta 100 razy w normalnych warunkach pracy systemu.
		\item Uruchomienie wyżej wspomnianego skryptu w warunkach awarii, czyli gdy działa jeden serwer warstwy zewnętrznej i jeden serwer warstwy zewnętrznej.   
	\end{enumerate}
	\item Stop usługi
	\begin{enumerate}[label*=\arabic*.]
		\item Wywołanie skryptu zamykającego usługę.
		\item Sprawdzenie zamknięcia na każdym węźle.  
	\end{enumerate}
\end{enumerate}
